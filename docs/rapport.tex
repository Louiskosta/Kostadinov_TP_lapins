\documentclass[12pt,a4paper]{article}

% Packages
\usepackage[utf8]{inputenc}
\usepackage[french]{babel}
\usepackage[T1]{fontenc}
\usepackage{graphicx}
\usepackage{geometry}
\usepackage{amsmath}
\usepackage{amssymb}
\usepackage{hyperref}
\usepackage{listings}
\usepackage{xcolor}
\usepackage{float}
\usepackage{caption}
\usepackage{subcaption}
\usepackage{fancyhdr}
\usepackage{titlesec}

% Géométrie de la page
\geometry{
    left=2.5cm,
    right=2.5cm,
    top=3cm,
    bottom=3cm
}

% Configuration des liens hypertexte
\hypersetup{
    colorlinks=true,
    linkcolor=blue,
    filecolor=magenta,
    urlcolor=cyan,
    citecolor=green,
    pdftitle={Rapport TP Simulation Population Lapins},
    pdfauthor={Votre Nom}
}

% Configuration du code source
\lstset{
    language=C,
    basicstyle=\ttfamily\small,
    keywordstyle=\color{blue}\bfseries,
    commentstyle=\color{green!60!black}\itshape,
    stringstyle=\color{red},
    numbers=left,
    numberstyle=\tiny\color{gray},
    stepnumber=1,
    numbersep=8pt,
    backgroundcolor=\color{gray!10},
    showspaces=false,
    showstringspaces=false,
    showtabs=false,
    frame=single,
    rulecolor=\color{black},
    tabsize=4,
    captionpos=b,
    breaklines=true,
    breakatwhitespace=false,
    escapeinside={\%*}{*)},
    xleftmargin=17pt,
    framexleftmargin=17pt,
    framexrightmargin=5pt,
    framexbottommargin=4pt
}

% En-têtes et pieds de page
\pagestyle{fancy}
\fancyhf{}
\fancyhead[L]{\leftmark}
\fancyhead[R]{TP Simulation}
\fancyfoot[C]{\thepage}
\renewcommand{\headrulewidth}{0.4pt}
\renewcommand{\footrulewidth}{0.4pt}

% Titre
\title{
    \vspace{-2cm}
    \textbf{Rapport de TP : Simulation de populations de lapins} \\
    \large Modélisation stochastique et analyse statistique
}
\author{
    Louis KOSTADINOV \\
    \textit{ZZ2 F2} \\
    \texttt{louis.kostadinov@etu.isima.fr}
}
\date{\today}

\begin{document}

\maketitle
\thispagestyle{fancy}

\begin{abstract}
Ce rapport présente une étude de simulation stochastique de populations de lapins. Le modèle implémenté en langage C intègre des mécanismes de reproduction et de vieillissement basés sur des probabilités, permettant d'analyser l'évolution démographique sur plusieurs années. Trois types d'expérimentations ont été menées : comparaison de conditions initiales, analyse de la variabilité, et étude statistique par boîtes à moustaches. Les résultats montrent une croissance exponentielle des populations avec une forte variabilité inter-réplication, caractéristique des processus stochastiques. L'utilisation de bibliothèques de précision arbitraire (GMP/MPFR) permet de gérer des populations dépassant plusieurs milliards d'individus.
\end{abstract}

\tableofcontents
\newpage

\section{Introduction}

\subsection{Contexte}
La dynamique des populations est un domaine fondamental de l'écologie mathématique. Ce travail pratique vise à étudier l'évolution d'une population de lapins à travers une simulation informatique intégrant des processus stochastiques de reproduction et de mortalité.

\subsection{Choix technologiques}
Ce projet a été développé en langage C pour plusieurs raisons :
\begin{itemize}
    \item \textbf{Efficacité d'exécution} : Le C offre des performances optimales pour les calculs intensifs
    \item \textbf{Efficacité mémoire} : Gestion fine de la mémoire, essentielle pour des populations gigantesques
    \item \textbf{Compréhension approfondie} : Le C permet de mieux comprendre le fonctionnement interne des algorithmes et des structures de données
\end{itemize}

\textit{Note : La documentation Doxygen, le fichier README et le Makefile ont été élaborés avec l'assistance de l'IA Claude (Anthropic).}

\subsection{Objectifs}
Les objectifs de ce TP sont multiples :
\begin{itemize}
    \item Implémenter un modèle de simulation démographique en langage C
    \item Intégrer des mécanismes probabilistes de reproduction et vieillissement
    \item Analyser l'influence des conditions initiales sur l'évolution des populations
    \item Étudier la variabilité inter-réplication des simulations stochastiques
    \item Produire des visualisations graphiques des résultats
\end{itemize}

\subsection{Organisation du rapport}
Ce rapport est organisé en plusieurs sections : la section~\ref{sec:methodo} décrit la méthodologie et le modèle mathématique, la section~\ref{sec:implementation} présente l'implémentation informatique, la section~\ref{sec:experiences} détaille les expérimentations menées, la section~\ref{sec:resultats} présente les résultats obtenus, et enfin la section~\ref{sec:conclusion} propose une discussion et une conclusion.

\section{Méthodologie} \label{sec:methodo}

\subsection{Modèle mathématique}

\subsubsection{Structure de la population}
La population est structurée selon l'âge et le sexe. Chaque individu est caractérisé par :
\begin{itemize}
    \item Son âge : défini par le mois et l'année de naissance (les lapins ne meurent pas forcément à 12 mois, ils grandissent d'une année en année)
    \item Son sexe : mâle ou femelle
\end{itemize}

On note $F_{m,a}(t)$ et $M_{m,a}(t)$ respectivement le nombre de femelles et de mâles nés au mois $m$ de l'année $a$, observés au temps $t$.

\subsubsection{Structures de données}
Le code utilise des structures de données minimalistes pour maximiser l'efficacité. Voici les définitions exactes utilisées dans l'implémentation :

\begin{lstlisting}[caption={Structure mois\_lapin - Représentation d'une classe d'âge}]
typedef struct mois_lapin {
    mpz_t nb_babies;                // Bebes (non-matures)
    mpz_t nb_male;                  // Males adultes
    mpz_t femelles_par_accouchements_restants[10]; // Femelles par nb portees
} mois_lapin;
\end{lstlisting}

\begin{lstlisting}[caption={Structure population - Organisation globale}]
#define AGE_MAX 16   // Age maximum en annees
#define NB_MONTHS 12 // Nombre de mois

typedef struct population {
    mois_lapin lapins_par_age[AGE_MAX + 1][NB_MONTHS];
} population;
\end{lstlisting}

\textbf{Caractéristiques clés :}
\begin{itemize}
    \item \textbf{Granularité temporelle} : Tableau 2D [année][mois] permettant de suivre précisément l'âge
    \item \textbf{Groupement par accouchements} : Les femelles sont regroupées par nombre d'accouchements restants dans l'année (0 à 9), évitant de stocker chaque individu séparément
    \item \textbf{Précision arbitraire} : Utilisation de \texttt{mpz\_t} (GMP) pour gérer des milliards d'individus
    \item \textbf{Compacité mémoire} : Pas de listes chaînées, uniquement des tableaux statiques et des compteurs
\end{itemize}

Cette conception minimaliste permet de gérer efficacement des populations de plusieurs milliards d'individus tout en maintenant la précision sur l'âge et le statut reproductif.

\subsubsection{Reproduction}
La reproduction suit un processus stochastique. Les probabilités sont calculées individuellement pour chaque lapin du même âge (même mois, même année, même sexe) :
\begin{itemize}
    \item Une femelle atteint la maturité sexuelle à 6 mois
    \item Pour chaque femelle mature, la reproduction suit une \textbf{loi de Bernoulli} si le nombre de femelles de cet âge est petit
    \item Pour un grand nombre de femelles du même âge, on utilise une \textbf{approximation par loi normale}, car les lois binomiales s'approximent en loi normale pour des grands nombres. Cela permet d'être très efficace pour des populations gigantesques
    \item Le nombre de lapereaux par portée suit une loi de probabilité donnée
    \item Le sexe de chaque lapereau est déterminé aléatoirement (probabilité 0.5)
\end{itemize}

Le seuil entre "petit nombre" et "grand nombre" est défini par la constante modifiable \texttt{GRAND\_NB = 10000} dans le code. Cette valeur peut être ajustée selon les besoins de précision et de performance.

Formellement, pour un groupe de $n$ femelles d'âge $(m,a)$ ayant une probabilité $p$ de reproduction :
\[
\text{Si } n < \texttt{GRAND\_NB} : N \sim \sum_{i=1}^{n} \text{Bernoulli}(p)
\]
\[
\text{Si } n \text{ grand : } N \sim \mathcal{N}(np, np(1-p))
\]

\subsubsection{Vieillissement et mortalité}
Le vieillissement est modélisé par :
\begin{itemize}
    \item Chaque mois, tous les individus vieillissent d'un mois
    \item Un taux de survie mensuel dépend de l'âge (mois et année)
    \item Comme pour la reproduction, la mortalité est calculée individuellement pour chaque lapin du même âge selon une \textbf{loi de Bernoulli} pour les petits nombres, ou une \textbf{approximation par loi normale} pour les grands nombres (modèle binomial)
\end{itemize}

L'équation de transition pour un groupe de lapins d'âge $(m,a)$ est :
\[
\text{Survivants}(m,a,t+1) = \text{Bernoulli}(s_{m,a}) \text{ ou } \mathcal{N}(n \cdot s_{m,a}, n \cdot s_{m,a}(1-s_{m,a}))
\]
selon la taille du groupe.

\subsection{Générateur de nombres aléatoires}
Le modèle utilise l'algorithme Mersenne Twister MT19937, un générateur pseudo-aléatoire de haute qualité avec une période de $2^{19937}-1$, garantissant une excellente distribution statistique.

\subsection{Précision numérique}
Pour gérer des populations très importantes (plusieurs milliards), le programme utilise :
\begin{itemize}
    \item \textbf{GMP} (GNU Multiple Precision) pour les entiers arbitrairement grands
    \item \textbf{MPFR} (Multiple Precision Floating-Point Reliable) pour les calculs en virgule flottante
\end{itemize}

\section{Implémentation} \label{sec:implementation}

\subsection{Architecture logicielle}

Le projet est organisé selon une architecture modulaire professionnelle :

\begin{verbatim}
TP_lapins/
  src/
    core/              Modules coeur de la simulation
      simulation.{c,h}    Moteur de simulation
      config.{c,h}        Configuration parametres
      population.{c,h}    Gestion populations
      aging.{c,h}         Vieillissement
      reproduction.{c,h}  Reproduction
    programs/          Programmes d'experimentation
      main.c             Programme principal
      fibo.c             Etude suite Fibonacci
      experiments.c      Experiences multiples
      graphiques.c       Generation graphiques
    external/          Bibliotheques externes
      mt19937ar-cok/     Generateur aleatoire
  bin/                 Executables compiles
  build/               Fichiers objets
  data/                Donnees et graphiques
  docs/                Documentation
\end{verbatim}

\subsection{Modules principaux}

\subsubsection{Module simulation}
Le module \texttt{simulation.c/h} constitue le cœur du système :
\begin{itemize}
    \item \texttt{simulate\_month()} : Simulation d'un mois (reproduction + vieillissement)
    \item \texttt{simulate\_year()} : Simulation d'une année complète
    \item \texttt{simulate\_population()} : Simulation sur N années
\end{itemize}

\subsubsection{Module reproduction}
Le module \texttt{reproduction.c/h} gère la reproduction :
\begin{itemize}
    \item Détection des femelles matures (âge $\geq$ 6 mois)
    \item Tirage aléatoire pour chaque femelle (reproduction ou non)
    \item Génération du nombre de lapereaux (loi de probabilité)
    \item Attribution aléatoire du sexe (mâle/femelle)
\end{itemize}

\subsubsection{Module aging}
Le module \texttt{aging.c/h} implémente le vieillissement :
\begin{itemize}
    \item Application des taux de survie par classe d'âge
    \item Décalage des classes d'âge (+1 mois)
    \item Élimination des individus de 12 mois et plus
\end{itemize}

\subsection{Système de build}

Le projet utilise GNU Make avec plusieurs cibles :
\begin{itemize}
    \item \texttt{make} : Compilation complète
    \item \texttt{make run-exe} : Exécution simulation de base
    \item \texttt{make run-graphiques} : Génération des graphiques
    \item \texttt{make doc} : Génération documentation Doxygen
    \item \texttt{make clean} : Nettoyage fichiers temporaires
\end{itemize}

\section{Expérimentations} \label{sec:experiences}

Trois types d'expérimentations ont été menées pour analyser le comportement du modèle.

\subsection{Expérience 1 : Comparaison des conditions initiales}

\subsubsection{Protocole}
\begin{itemize}
    \item \textbf{Objectif} : Étudier l'influence de la population initiale
    \item \textbf{Conditions testées} :
    \begin{itemize}
        \item Condition A : 10 mâles, 10 femelles (total 20)
        \item Condition B : 50 mâles, 50 femelles (total 100)
        \item Condition C : 100 mâles, 100 femelles (total 200)
    \end{itemize}
    \item \textbf{Durée} : 15 ans (180 mois)
    \item \textbf{Réplications} : 1 simulation par condition
    \item \textbf{Mesure} : Population totale mensuelle
\end{itemize}

\subsubsection{Hypothèses}
On s'attend à observer une relation proportionnelle entre la taille initiale et la croissance finale, avec une possible croissance exponentielle caractéristique des modèles de populations non limitées.

\subsection{Expérience 2 : Analyse de la variabilité}

\subsubsection{Protocole}
\begin{itemize}
    \item \textbf{Objectif} : Quantifier la variabilité inter-réplication
    \item \textbf{Condition} : 100 mâles, 100 femelles
    \item \textbf{Durée} : 15 ans (180 mois)
    \item \textbf{Réplications} : 10 simulations indépendantes
    \item \textbf{Mesure} : Population totale mensuelle pour chaque réplication
\end{itemize}

\subsubsection{Hypothèses}
La nature stochastique du modèle devrait générer une variabilité importante entre les réplications, potentiellement croissante avec le temps.

\subsection{Expérience 3 : Étude statistique (boxplots)}

\subsubsection{Protocole}
\begin{itemize}
    \item \textbf{Objectif} : Analyse statistique comparative approfondie
    \item \textbf{Conditions testées} : 3 (comme exp. 1)
    \item \textbf{Durée} : 12 ans (144 mois)
    \item \textbf{Réplications} : 15 simulations par condition
    \item \textbf{Mesure} : Distribution des populations annuelles
    \item \textbf{Visualisation} : Boîtes à moustaches (boxplots)
\end{itemize}

\subsubsection{Analyse statistique}
Pour chaque condition et chaque année, on calcule :
\begin{itemize}
    \item Médiane : $Q_2$ (50\textsuperscript{e} percentile)
    \item Premier quartile : $Q_1$ (25\textsuperscript{e} percentile)
    \item Troisième quartile : $Q_3$ (75\textsuperscript{e} percentile)
    \item Intervalle interquartile : $IQR = Q_3 - Q_1$
    \item Valeurs extrêmes (outliers)
\end{itemize}

\section{Résultats} \label{sec:resultats}

\subsection{Expérience 1 : Comparaison des conditions initiales}

\begin{figure}[H]
    \centering
    \includegraphics[width=0.95\textwidth]{../data/comparaison_populations.png}
    \caption{Évolution des populations pour trois conditions initiales différentes sur 15 ans}
    \label{fig:comparaison}
\end{figure}

La figure~\ref{fig:comparaison} montre l'évolution temporelle des populations pour les trois conditions initiales.

\subsubsection{Observations}
\begin{itemize}
    \item \textbf{Croissance exponentielle} : Les trois courbes présentent une allure exponentielle caractéristique
    \item \textbf{Influence de la condition initiale} : 
    \begin{itemize}
        \item Condition A (10×10) : Population finale $\approx$ X individus
        \item Condition B (50×50) : Population finale $\approx$ Y individus
        \item Condition C (100×100) : Population finale $\approx$ Z individus
    \end{itemize}
    \item \textbf{Séparation des courbes} : Les trajectoires restent distinctes, suggérant une dépendance persistante aux conditions initiales
    \item \textbf{Phase de latence} : Une courte période initiale de croissance lente est observable
\end{itemize}

\subsubsection{Interprétation}
La croissance exponentielle observée est cohérente avec un modèle sans limitation de ressources (pas de capacité de charge). Le taux de croissance semble constant après une phase transitoire initiale.

\subsection{Expérience 2 : Analyse de la variabilité}

\begin{figure}[H]
    \centering
    \includegraphics[width=0.95\textwidth]{../data/variabilite_populations.png}
    \caption{Variabilité inter-réplication pour 10 simulations indépendantes (100×100 initial)}
    \label{fig:variabilite}
\end{figure}

La figure~\ref{fig:variabilite} présente les 10 trajectoires individuelles obtenues avec les mêmes paramètres mais des graines aléatoires différentes.

\subsubsection{Observations}
\begin{itemize}
    \item \textbf{Variabilité croissante} : L'écart entre les réplications augmente avec le temps
    \item \textbf{Tendance commune} : Toutes les réplications suivent une tendance exponentielle similaire
    \item \textbf{Divergence tardive} : La variabilité devient très importante après 10-12 ans
    \item \textbf{Aucune extinction} : Aucune simulation ne conduit à l'extinction (population > 0)
\end{itemize}

\subsection{Expérience 3 : Étude statistique (boxplots)}

\begin{figure}[H]
    \centering
    \includegraphics[width=0.95\textwidth]{../data/boxplot_populations.png}
    \caption{Distributions annuelles des populations pour trois conditions initiales (15 réplications × 3 conditions)}
    \label{fig:boxplot}
\end{figure}

La figure~\ref{fig:boxplot} présente les distributions de populations sous forme de boîtes à moustaches pour chaque condition et chaque année.

\subsubsection{Observations}
\begin{itemize}
    \item \textbf{Médiane croissante} : La médiane augmente exponentiellement pour toutes les conditions
    \item \textbf{IQR croissant} : L'intervalle interquartile s'élargit avec le temps, confirmant l'augmentation de la variabilité
    \item \textbf{Asymétrie} : Les distributions deviennent de plus en plus asymétriques (queue supérieure allongée)
    \item \textbf{Outliers} : Présence possible de valeurs extrêmes, surtout pour les temps tardifs
    \item \textbf{Séparation des conditions} : Les trois conditions restent statistiquement distinctes
\end{itemize}

\subsubsection{Comparaison statistique}
Les boîtes à moustaches permettent de comparer visuellement :
\begin{itemize}
    \item La tendance centrale (médiane) entre conditions
    \item La dispersion (IQR) entre conditions
    \item La présence d'observations atypiques
\end{itemize}

Un test statistique (par ex. Kruskal-Wallis) pourrait confirmer la différence significative entre les trois conditions pour chaque année.

\section{Discussion}

\subsection{Validation du modèle}

\subsubsection{Cohérence biologique}
\begin{itemize}
    \item La croissance exponentielle est réaliste pour une population sans limitation de ressources
    \item Les taux de reproduction observés sont cohérents avec la biologie des lagomorphes
    \item L'absence d'extinction indique des paramètres de survie et reproduction favorables
\end{itemize}

\subsubsection{Validation numérique}
\begin{itemize}
    \item L'utilisation de GMP/MPFR garantit la précision des calculs
    \item Le générateur MT19937 assure une bonne qualité des tirages aléatoires
    \item La modularité du code facilite la vérification et le débogage
\end{itemize}

\subsection{Limites du modèle}

\subsubsection{Limitations biologiques}
\begin{itemize}
    \item \textbf{Absence de capacité de charge} : Pas de limitation par la nourriture disponible. Dans la réalité, les ressources alimentaires limitent la croissance des populations
    \item \textbf{Pas de prédation} : Aucun facteur de mortalité externe
    \item \textbf{Reproduction simplifiée} : Pas de saisonnalité, consanguinité, etc.
    \item \textbf{Environnement idéal} : Nourriture illimitée, absence de maladies, conditions climatiques optimales
\end{itemize}

\subsubsection{Limitations numériques}
\begin{itemize}
    \item Les calculs correspondent à un modèle binomial pour les grands nombres, nécessitant l'approximation par la loi normale pour maintenir l'efficacité
    \item La variabilité stochastique rend difficile la prédiction précise
\end{itemize}

\section{Conclusion} \label{sec:conclusion}

Ce travail pratique a permis de développer et d'analyser un modèle stochastique de dynamique de population de lapins. L'implémentation en langage C, avec une architecture modulaire professionnelle, garantit la maintenabilité et l'extensibilité du code.

\subsection{Résultats principaux}
\begin{enumerate}
    \item Le modèle produit une croissance exponentielle caractéristique des populations non limitées
    \item Les conditions initiales influencent l'amplitude mais pas la forme de la croissance
    \item La variabilité inter-réplication augmente avec le temps, reflétant la nature stochastique du processus
    \item Les analyses statistiques (boxplots) confirment la robustesse des tendances observées
\end{enumerate}

\subsection{Compétences acquises}
\begin{itemize}
    \item Implémentation d'un modèle stochastique en C
    \item Utilisation de bibliothèques de précision arbitraire (GMP/MPFR)
    \item Génération de nombres pseudo-aléatoires de qualité (MT19937)
    \item Visualisation de données avec Gnuplot
    \item Documentation automatique avec Doxygen
    \item Gestion de projet avec Make et Git
    \item Analyse statistique de simulations
\end{itemize}

\subsection{Performance et limites physiques}

La combinaison d'une structure de données minimale et de l'approximation par loi normale permet d'atteindre un très grand nombre d'années de simulation très rapidement. L'utilisation de \texttt{mpz\_t} (GMP) autorise la manipulation de nombres arbitrairement grands, rendant possible la simulation de 1000 ans ou plus sans overflow numérique.

Cependant, à de telles échelles temporelles, les nombres obtenus deviennent tellement astronomiques qu'ils dépassent toute réalité physique : on ne peut plus dire que le nombre de lapins est simplement "très grand", car l'Univers observable lui-même ne serait pas assez vaste pour contenir une telle population. Cela illustre les limites d'un modèle de croissance exponentielle sans contraintes de ressources.

\appendix

\section{Code source}

\subsection{Exemple : Fonction de simulation d'un mois}

\begin{lstlisting}[caption={Extrait du module simulation.c - Fonction simulate\_month()}]
void simulate_month(population *pop) {
    // 1. Reproduction
    // Ajoute les nouveaux-nes (age 0) selon les regles
    reproduction_month(pop);
    
    // 2. Vieillissement
    // Vieillit tous les lapins d'un mois
    aging_month(pop);
}
\end{lstlisting}

\subsection{Exemple : Fonction de génération de graphiques}

\begin{lstlisting}[caption={Extrait du module graphiques.c - Génération d'un graphique}]
void graphique_comparaison_simple() {
    // Simulations pour 3 conditions initiales
    for (int cond = 0; cond < 3; cond++) {
        int init_m = initial_males[cond];
        int init_f = initial_females[cond];
        
        population *pop = create_population();
        init_population(pop, init_m, init_f);
        
        // Simulation sur 15 ans
        for (int year = 0; year <= 15; year++) {
            // Enregistrement donnees
            save_data(pop, year, cond);
            
            if (year < 15) {
                simulate_year(pop);
            }
        }
        
        free_population(pop);
    }
    
    // Generation du graphique avec Gnuplot
    generate_gnuplot_script();
}
\end{lstlisting}

\section{Références}

\begin{thebibliography}{99}

\bibitem{caswell2001}
Caswell, H. (2001). \textit{Matrix Population Models: Construction, Analysis, and Interpretation}. Sinauer Associates, 2nd edition.

\bibitem{gotelli2008}
Gotelli, N.J. (2008). \textit{A Primer of Ecology}. Sinauer Associates, 4th edition.

\bibitem{matsumoto1998}
Matsumoto, M., \& Nishimura, T. (1998). Mersenne Twister: A 623-dimensionally equidistributed uniform pseudo-random number generator. \textit{ACM Transactions on Modeling and Computer Simulation}, 8(1), 3-30.

\bibitem{gmp}
GNU Multiple Precision Arithmetic Library. \url{https://gmplib.org/}

\bibitem{mpfr}
The MPFR Library. \url{https://www.mpfr.org/}

\bibitem{williams2007}
Williams, B.K., Nichols, J.D., \& Conroy, M.J. (2007). \textit{Analysis and Management of Animal Populations}. Academic Press.

\end{thebibliography}

\end{document}
